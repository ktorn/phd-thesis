\chapter{Conclusion}


boilerplate text, boilerplate text, boilerplate text, boilerplate text, boilerplate text, boilerplate text, boilerplate text, boilerplate text, boilerplate text, boilerplate text.



\section{Recommended Best Practices}


Defensive programming (fallbacks) - always render something
Clear indication on console.log when something has gone wrong (ideally following some standard)
Use UI overlays (menus) to allow users to re-configure API endpoints (ideally these configs stored in browser local storage)
include in OBJKT assets a directory: doc (which contains:
								technical info: native resolution, other supported resolutions and screen orientations (landscape, portrait)
								documentation about the artists' intention, and samples of network calls and expected responses)







\section{Future Work}


An idea is to work out a standard "introduction to smart contracts" academic programme, that is both agnostic of underlying platform but introduces students to the pros and cons of each existing platform.

Test citation to new ref  \cite{dekkerCollectingConservingNet2018}.

boilerplate text, boilerplate text, boilerplate text, boilerplate text, boilerplate text, boilerplate text, boilerplate text, boilerplate text, boilerplate text, boilerplate text.


\subsection{Other Classifications}

Should try to classify code-base artworks according to:

* Asset storage (private servers, IPFS, Arweave)
* Libraries/Dependencies
* Networked Dependencies (type, blockchain indexer vs other APIs, public endpoints or private)
* Open source or not (obfuscated code, code running on private infra)



\subsection{Survey and Classification of Interactivity Types}

As discussed in \autoref{sec:interactivity} and \autoref{subsec:capture-interactive}, there is a wide range of interactivity types in code-base artworks, each posing different preservation requirements, so it is important to identify and classify these artworks. Future work will survey the artifacts and attempt to classify them using a mixture of static code analysis, and analysis of screenshots by LLMs.


\subsection{Personal Curation Logs}

mention fxhash articles (https://docs.fxhash.xyz/articles) and their tezos pointer scheme (https://github.com/fxhash/specifications/blob/main/general/tezos-storage-pointers.md) which may be useful for the idea of cross contract references, which is my plan for documentation (each person, one contract, cross-referecning OBJKTs)

\subsection{Support for WACZ file export}

The choice not to pursue webrecorder snapshots for this work was purely pragmatic, given the issues in making the programatic snapshots work and the time constraints of the project. However it would be beneficial for Arkivo snapshots to be exportable as WACZ files, as it is a well established standard which would add value to the data recorded by Arkivo. Therefore support for WACZ snapshots/exports has been added to the short-term roadmap for Arkivo.

\subsection{Risk Analysis}

Mention other risk analysis tools \cite{l2beatL2BEATStateLayer2024}


\subsection{Security: Detecting and Dealing with Malicious Code}

In this section I will talk about the security aspects of code-base artworks.





\subsection{Copyright Measures}


As discussed in \autoref{subsec:chap2_copyright} - \nameref{subsec:chap2_copyright} , page \pageref{subsec:chap2_copyright}... 