\chapter{Conclusion}

This chapter concludes the study by reviewing what was accomplished and how it relates to the initial goals of the project. It also describes the main ideas reserved for future work.

\section{Review of Objectives}

As stated in chapter 1, this work had 2 mains objectives:

\begin{itemize}
	\item To design and implement a sustainable IS that can support the long-term preservation of networked crypto art
	\item To contribute to the body of knowledge in the area of networked crypto art conservation
\end{itemize}

These objectives were then broken down into 4 research questions:

\begin{itemize}
	\item RQ1 - Can we determine which code-based HEN OBJKTs are networked?
	\item RQ2 - Can we capture and document networked HEN OBJKTs aesthetic evolution?
	\item RQ3 - What are the disk space requirements for storing all code-based HEN OBJKTs' media assets?
	\item RQ4 - Can an online archive for HEN OBJKTs be economically self-sustainable?
\end{itemize}

With the development or \texttt{ARKIVO} we were able to answer these questions:

RQ1 - \texttt{ARKIVO} is able to classify which code-based artworks are networked, and as far as we could test it has a perfect success rate (100\%).

RQ2 - the snapshot feature in \texttt{ARKIVO} can capture an OBJKT's aesthetic evolution, and in the process document it. A question remains about the disk space requirements for regular snapshots, however with the introduction of the Home Nodes, this problem may already be solved

RQ3 - this question was answered in \autoref{sub:rq1_mile3} (\nameref{sub:rq1_mile3}, page \pageref{sub:rq1_mile3}): 79.3GB space on disk.

RQ4 - this question is the most challenging, because to be economically sustainable the platform would need to break even financially, yet currently it has no sources of funding. The effort to decentralise the snapshots by a network of home nodes should go a long way towards achieving this goal, but the project also needs to receive a regular source of funding, and the concept of royalty splits contributing to the project, in a VDP style, is something we would like to encourage.

Archiving all code-based HEN OBJKTs at a monthly cost of EUR 15.90 provides a useful data point on the sustainability of small scale digital archives, and is one of the contributions of this project to the body of knowledge.



\section{Recommended Best Practices}

When creating a networked crypto artwork, artists should have in mind the following best practices:

 \begin{itemize}
	\item Use defensive programming techniques, so that your code does not fail in unexpected ways. Add fall-back mechanisms at several key points so that the piece will degrade gracefully rather than break unexpectedly;
	\item In the worst case scenario, when all else fails, have one last fall-back mechanism so that the piece renders a static version that is aesthetically pleasing;
	\item When an error does occur, make it clear on the console log.
	\item Add an overlay menu that the user can pull up to re-configure external dependency URLs. Save these settings in Local Storage so that the user does not have to re-input them every time they access the piece;
	\item Add documentation within the artwork's assets directory. This should be in a plain-text document, describing the piece in detail. Technical details such as native display resolution, orientation, etc. Also it should include samples of expected data returned by network dependencies and how they are used by the artwork;
 \end{itemize}


\section{Future Work}

This work laid the foundation for future work in the area of code-based crypto art preservation. Some of the ideas for such work are as follows:

\begin{itemize}
	\item Study methods of automatically classifying the artworks according to the taxonomy proposed in \autoref{sec:interactivity} (page \pageref{sec:interactivity}). This work could potentially employ the use of Large Language Models (LLMs);
	\item Implement a thorough security scan of the artwork's code, using a variety of methods, as described in \autoref{sub:codesec} (\nameref{sub:codesec}, page \pageref{sub:codesec});
	\item Integrate the WARC file format into the snapshot feature, so that snapshots comply with this industry standard;
	\item Allow users to login with their crypto wallets so as being able to experience owner specific versions of the artworks;
	\item Create a staging area for collaborative documentation, between artist and curator, which can be published to the blockchain when finalized;
	\item Create an area for restoration of broken artworks;
\end{itemize}


\section{Personal Remarks}

\texttt{ARKIVO} is very different from existing digital art archives in the sense that it is currently an automated platform. There is no manual acquisition of artworks, nor curation of listed OBJKTs. Every newly minted code-based OBJKT is automatically indexed and added to the archive. This was intentional, as it contributes to a lower cost of running the platform. This approach is not meant to diminish the role of human conservators. Much the opposite. In its limitations it highlights the important role that human conservators play in the preservation of digital art.


This study aimed to deal with the mutable dependency dilemma, by creating a way to document the evolution of artworks which are, a priori, destined to break. Whilst this is a worthwhile endeavour, another approach may be to avoid the dilemma altogether and embrace mutability from the outset. The blockchain's history may be immutable, but the chain itself is changing with each new block. Why not adopt the idea of a mutable artwork, which can not only be patched, but which can be even more fundamentally evolved by its creator? This is a conversation that needs to take place within the crypto art community.

