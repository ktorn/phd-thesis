\chapter{Discussion}


boilerplate text, boilerplate text, boilerplate text, boilerplate text, boilerplate text, boilerplate text, boilerplate text, boilerplate text, boilerplate text, boilerplate text.


\section{Cost of Storage}

Currently, the cost of minting an OBJKT on the HEN contract constitutes a fraction of the cost of hosting and/or pinning 2GB of data. This represents a loophole that could be exploited by malicious actors, or freeloaders, making the pinning of assets hosted by Teia unfeasible without a significant recurring investment, which impacts the economic sustainability of the project.

\section{Emulation}

As time passes, technology will continue to evolve. New hardware will run new operating systems, which will run new web browsers, or even totally paradigm changing \emph{content rendering applications} which may replace web browsers in the future. This means that in long term it seems inevitable that all code-base artworks created today, even those based on web standards and without any network dependencies, will require some form of hardware and/or browser emulation to render successfully. 

This inevitable evolution into emulated environments does represent an additional challenge for networked artworks, because in addition to emulating the artefact's code-base itself, the emulation environment must also simulate the network dependencies of the artwork, and this can pose a problem known to software developers as \emph{dependency hell}, where each dependency in turn depends on a number of other dependencies, leading to an ever increasing dependency tree. A solution to this problem seems unfeasible, if any of those dependencies, or their sub-dependencies, does not strictly follow the web3 architecture that makes projects easily duplicable.

For this reason, the IOC model gains an even greater importance for the longevity of code-based networked artworks, because you can more easily generate the data expected by the artwork, or in the worst-case scenario, simply repeatedly inject the last known good snapshot of data for all future renderings.

\section{Beyond the Canvas}

All of the artworks which this project currently aims to cater for in terms of conservation have one thing in common: their visual rendering is restricted to the boundaries of an HTML element on a web page, the canvas. Even those which intentionally choose to spread their visual expression past the boundaries of a single canvas element, are still constrained by the larger HTML DOM object or web page in which they, as web-based artworks, essentially exist. This is not unique to digital art. The great majority of paintings displayed in traditional museums are also constrained by their physical framed canvasses. Sculptures and other art installations are exceptions to this rule. However networked art reaches out into the world, and it would seem appropriate to explore a rendering medium that represents that breaking away from the canvas.