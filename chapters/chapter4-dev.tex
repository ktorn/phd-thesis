\chapter{Prototype Development}


boilerplate text, boilerplate text, boilerplate text, boilerplate text, boilerplate text, boilerplate text, boilerplate text, boilerplate text, boilerplate text, boilerplate text.


\section {Prototype Requirements}

\subsection{Risk Analysis}

Mention other risk analysis tools \cite{l2beatL2BEATStateLayer2024}

TODO:
* Asset storage (private servers, IPFS, Arweave)
* Libraries/Dependencies
* Networked Dependencies (type, blockchain indexer vs other APIs, public endpoints or private)
* Open source or not (obfuscated code, code running on private infra)




\section {Prototype 1 - genartdex}

\section{System Architecture}

boilerplate text, boilerplate text, boilerplate text, boilerplate text, boilerplate text, boilerplate text, boilerplate text, boilerplate text, boilerplate text, boilerplate text.

\subsection{Infrastructure}

A typical centralised application would be sensible to settle for state-of-the-art deployment platforms like Amazon ACS, DigitalOcean, Heroku, or Microsoft Azure. Not only do these commercial infrastructure solutions offer reliable up-time with fast Internet access, but also a wide range of geographically dispersed data centres across the world for maximum resiliency. It is theoretically possible to deploy a distributed and decentralised (control) solution across these platforms from an application point of view, however at the infrastructure level, the control would be centralised on a small number of industry players, which are an easy target for censorship requests from governments (ref needed).

\subsection{Smart Contracts}

boilerplate text, boilerplate text, boilerplate text, boilerplate text, boilerplate text, boilerplate text, boilerplate text, boilerplate text, boilerplate text, boilerplate text.

\subsection{Authentication}

todo: authentication vs authorisation vs access control

One of the main benefits of web3-style authentication is that it does not require any user credentials stored in the application's database. Since all web3 users have their public keys publicly registered on the blockchain, any user can identify themselves by \emph{connecting} their web3 wallet and signing a custom message provided by the application. This way the application can verify who the user is and that they are in possession of the private key, therefore validating their authentication credentials.

\section{Open Source}

Open sourcing this project was essencial because...

\subsection{Open Source License}

The choice of license...


The choice of license for this project was ... due to...



\section{Security: Detecting and Dealing with Malicious Code}

In this section I will talk about the security aspects of code-base artworks.