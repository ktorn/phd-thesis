\chapter*{Appendix 2}
\label{appx:chainleft-taxonomy}

\section*{Chainleft's Taxonomy Descriptions}

\addcontentsline{toc}{chapter}{Appendix 2 - Chainleft's Taxonomy Descriptions}



Full descriptions of Chainleft's Taxonomy, provided by private message in Discord. Included with permission.


\begin{quote}
``\textbf{Auto-Evolving}

Since EVM can run and calculate data constantly, these calculations can evolve the art itself. The parameters that create the artwork iterate over time to result in a more complex pattern. The difficulty in achieving this comes from new layers of parameters for the evolution that would aesthetically or conceptually still please \& produce qualia.

\textbf{Ever-Changing}

We could probably combine the evolution and ever-changing as a single category. However while evolutions tend to reach a level where the art’s final goal is fulfilled, ever-changing runtime art fully explores the foreverness of the blockchain runtime. The calculations simply never end.

\textbf{Programmable}

So far we’ve looked at how the virtual machine’s ceaseless work carries the art forward, but what about human agency? This is where programmability comes in. Several runtime art collections allow their collectors (and sometimes other agents!) to influence the artwork. This is sometimes within the confines of the limits that the artist determines, and sometimes complete freedom where the new work replaces the old one fully.

\textbf{Social Coordination}

If there’s one thing that’s very blockchain-native, that’s coordination. Even the validators that produce the blocks coordinate to make this system live. So some artists used social coordination mechanisms through smart contracts that resulted in artwork being produced through the activities of their collectors.

This is a relatively less explored sub-niche of runtime art, but I think it can get quite exciting when combined with the future of gaming NFTs.

\textbf{Ownership Aesthetics}

Ownership has been a major concept in the NFT space historically, mainly because of the trustless ownership advantages NFTs provide to collectors. Artists certainly tapped into this concept, particularly around “tokenness”.

Ownership-based Runtime Art takes this concept and forms aesthetics through it. Since smart contracts can access collectors’ balances of different coins as well as other NFTs, we can build art around this information. Artists can even offer unique on-chain traits to their collectors based on ownership of their prior work; which would go away if the work is transferred away.'' \cite{chainleftOnchainRuntimeArt2024}

\end{quote}