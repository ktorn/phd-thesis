\chapter*{Appendix 1 - Publications}
\addcontentsline{toc}{chapter}{Appendix 1 - Publications}
\label{chap:publication}





Farinha, D. F. (2022). A Critical Look at Blockchain-Interactive Generative Art. \emph{International Journal of Creative Interfaces and Computer Graphics (IJCICG)}, 13(1), 1-17.

\section*{Abstract}

Artists are increasingly using blockchain as a tool for trading digital artwork as non-fungible tokens (NFTs), however some are also beginning to experiment with the blockchain as a medium for generative art, using it as a seed for a generative process or to continuously modify an evolving piece. This paper surveys, reviews and classifies the state-of-the-art in blockchain-interactive NFTs and presents a liberal-arts critique of the opportunities and threats posed by this technology, whilst addressing existing criticism on the broader topic of art-related NFTs. The paper examines some of the most experimental pieces minted on the Hic et Nunc (HEN) and Teia NFT marketplaces, for which a purpose-built research tool was developed. The survey reveals some reliance on centralised infrastructure, namely blockchain indexers, placing undesired trust on third parties which undermines the potential longevity of the artwork. The paper concludes with recommendations for artists and NFT platform designers for developing more resilient and economically sustainable architectures.


\clearpage


\noindent Estadieu, G. V., Ng, S. O. K. M., Martins De Abreu, F., \& Farinha, D. (2023, November). Beyond Physical Boundaries-Organising a Virtual Exhibiton with NFTs for an International Conference. In \emph{Proceedings of the 11th International Conference on Digital and Interactive Arts} (pp. 1-5).

\section*{Abstract}

The COVID-19 pandemic has challenged individuals and all international collaborations, including research and academic international conferences. Most conferences had to adapt to an online or hybrid mode at best, if not cancelled or postponed. During this period, we were invited to organise the third edition of an international conference on Arts, Technology and Society. At its core, it is an academic conference, but it also offers a platform for an art exhibition and forum for artists to share and exhibit. Due to Macao's physical constraints and its inaccessibility for outside visitors, the Local Organising Committee took the challenge to organise the first virtual art exhibition in an academic conference, including Non-Fungible Tokens (NFT) for artists willing to sell their digital artwork. This paper presents the creative process as well as the technical issues faced during the conference and exhibition preparation time. As NFT-based virtual exhibition is a new offer within an academic international conference and a new tool to permanently promote artists’ work, we are also proposing some possible future works to develop further this approach and give it a larger visibility. We believe such type of exhibition can play an important role on a world seeking for more sustainable communication.