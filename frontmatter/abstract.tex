\chapter*{Abstract}
\addcontentsline{toc}{chapter}{Abstract}




Art registered on a blockchain as non-fungible tokens (NFTs), also known as crypto art, has gained popularity in the digital art world, because it enabled artists to create scarce editions of their work. This scarcity of a digital asset in turn bestows the artwork with economic value, which attracts collectors looking to invest in digital art, and thus allows for the emergence of a global market for digital art. Perhaps more interestingly from a cultural perspective, is that artists have also adopted the blockchain as an art medium. Similarly to net art in 1990s, some forms of crypto art are native to the blockchain and cannot conceptually exist without it. Code-based crypto art, and specifically networked artworks, can be conceptually interpreted as being self-aware and evolving. However these artworks are highly susceptible to obsolescence and, from a conceptual and technological perspective, pose significant challenges for their long-term preservation.
This study focused on understanding these challenges and on experimenting with various conservation techniques aimed at solving them. Following a Design Science Research methodology, this project produced a software artifact, \texttt{ARKIVO}, for automating the archival, storage, and documentation of some of these artworks. It also proposes a new paradigm for creating crypto art, which embraces mutability and evolution as its core principles.
